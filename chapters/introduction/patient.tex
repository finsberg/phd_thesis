\section{Patient Specific Model Generation}

In this section we will describe 

\subsection{Geometry and microstructure}

The geometry of a patient's heart can be aquired using medical
imaging techniques, such as echocardiography (ultracound), magnetic
resonance imaging (MRI) or computed tomography (CT). Each modality off
advantages and disadvatages over the other. For example, MRI provides
high quality images, uses zero radiation, but is exspensive and lacks
temporal resolution. CT can more accuratley reconstruct the 3D image
in contrast to MRI, in which 2D slices needs to be glued together to
form a 3D surface. However, CT exposes the patient to radiation which
increases the chance of developing cancer. Finally echocardiography is
easy to use, cheap, harmless, and  has good temporal resolution, but
is clearly inferior when it comes to image quality.

The main modality used in this thesis is 3D echocardiography



\begin{itemize}
  \item Segmentation and aquisition using echopach
  \item From closed endo- and epicardial surfaces to a mesh
    (smoothing, cutting, orienting).
  \item Marking of mesh accoring to AHA segments (16,17,18)
  \item Local basis
  \item Rule based fibers
  \item Referece to mesh{\_}generation and patient repositories
\end{itemize}

\subsection{Continuum description of the heart}

\begin{itemize}
  \item Reference to mehcanics book
  \item Introduce basic notation (stress, strain, deformation)
\end{itemize}

We represent the heart as a continuum body  embedded in
$\mathbb{R}^3$. Let $\Xvec$ and  $\xvec$ denote the corresponding
coordinate in the reference and current configuration respectively.
The motion of a point $\Xvec$ in the reference configuration to the
point $\xvec$ in the current configuration  may be described by a
deformation map  $\varphi :  \mathcal{B}_0  \rightarrow \mathcal{B}$.
The deformation gradient associated with the motion $\xvec =
\varphi(\Xvec)$ is a rank-2 tensor given by  
\begin{align}
\F(\Xvec) = \frac{\partial \varphi}{\partial \Xvec}, 
\end{align}
The deformed and reference configuration are related via the
displacement field $\uvec$ by: 
\begin{align}  
\uvec = \xvec-\Xvec = \chi( \Xvec, t) -\Xvec,
\end{align} 
Further, we let $J = \det \F$ be the determinant of the deformation gradient 
and $\C = \F^T\F$ the right Cauchy-Green deformation tensor.\cite{holzapfel2000nonlinear}


\subsection{Passive behavior of the myocardium}

\begin{itemize}
  \item Hyperelasticiy, fung type-refer to relevant papers
  \item Incompressibility - Saddle point problem
  \item Finding the unloaded configuration
\end{itemize}



\subsection{Active behavior of the myocardium}


\begin{itemize}
  \item Active stress and strain (some mathematical aspects perhaps)
\end{itemize}




\subsection{Adjoint-based Data Assimilation}

\begin{itemize}
  \item Examples of data you want to constrain
  \item Forming of mismatch functional and problem formulation as
    PDE-constrained optimization.
  \item The adjoint approach. comparison with other methods.
  \item Control parameters
  \item Reference to pulse{\_}adjoint and pulse{\_}adjoint{\_}post
    repositories
\end{itemize}


We will now explain what we mean by ``adjoint-based'', and why this
approach is a key ingredient. The main theory presented here is taken
from the dolfin-adjoint web page. A word about dolfin adjoint...\ref{}
  
\begin{equation}
  \begin{aligned}
    \label{eq:opt_matparam}
    & \underset{m}{\text{minimize}}
    & &  I(\state, m) \\
    & \text{subject to}
    & & \delta \Pi(\state, \mvec) = 0, 
  \end{aligned}
\end{equation}

where $I(\state, m): \mathbb{W} \times \mathbb{Q} \mapsto
\mathbb{R}$ for some state space $\mathbb{W}$ and parameter space
$\mathbb{Q}$ and $\delta \Pi(\state) = 0$ is the force balance
equation given by \ref{}.


In order to apply optimisation algorithm...
We reduce the objective functional to be a function of the control
parameters only, $\hat{I}(\mvec) := I(\state(\mvec), \mvec)$. Consider an
initial value of the parmameters $\mvec = \mvec^0$. If $\hat{I}(m)$ is defined
and differentible in a neighborhood of $\mvec^0$ then $\hat{I}$ (and
consequently $I$) decreases fastest in the direction of the gradient, 
\begin{align}
  \nabla_{\mvec} \hat{I}(\mvec^0) = \begin{bmatrix}
    \frac{\mathrm{d} \hat{I}(\mvec^0) }{\mathrm{d} m_1},
    \frac{\mathrm{d} \hat{I}(\mvec^0) }{\mathrm{d} m_2},
    \cdots
    \frac{\mathrm{d} \hat{I}(\mvec^0) }{\mathrm{d} m_N}
  \end{bmatrix}^T.
  \label{eq:functional_gradient}
\end{align}
Here $\frac{\mathrm{d} \hat{I} }{\mathrm{d} m_i}$ represents the
total derivative:
\begin{align}
  \frac{\mathrm{d} \hat{I} }{\mathrm{d} m_i} = \frac{\mathrm{d} I (\state(\mvec), \mvec))}{\mathrm{d} m_i} = \frac{\partial  I }{\partial \state} \frac{\mathrm{d} \state}{\mathrm{d} m_i} + \frac{\partial  I }{\partial m_i}.
  \label{eq:functional_derivative_component}
\end{align}
In other words, the sequence
\begin{align}
  \mvec^{k+1} = \mvec^{k} - \gamma_k \nabla_{\mvec} \hat{I}(\mvec^k), \gamma_n \in \mathbb{R}
\end{align}
satisfies $\hat{I}(\mvec^{k+1}) \leq \hat{I}(\mvec^k) \; \forall k \geq
0$, and converges towards a local minimum. If $\hat{I}$ is convex then
the minimum is also global.
Being able to compute the gradient of the objective functional wrt to
the control parameters allows us to employ gradient based optimization
methods which are in general superior to gradient free methods.
One way to compute the gradient is by means of the finite difference
approach: For a given parameter $\mvec = \mvec^*$ we have
\begin{align}
  \frac{\mathrm{d} \hat{I} }{\mathrm{d} m_i}( \mvec^*) =
  \lim_{h \mapsto 0} \frac{\hat{I}(\mvec^* + h\mathbf{e}_i) - \hat{I}(\mvec^*)}{h}, 
\end{align}
where $\mathbf{e}_i \in \mathbb{Q}$ is the $i$'th canonical basis
vector. If $\dim(\mathbb{Q}) = N$, this approach would require $N+1$
functional evaluations. Moreover, since the state-variables depends upon the
control variables, we would also need to solve the force balance
equation $N+1$ times, which is typically very copmutationally
expensive. Hence, this approach is typically infeasable when the
dimesion of your parameterspace is large. In this case the adjoint
approach is much better. If $A$ is an operator (e.g a matrix), then
the adjoint operator $B$ satisfies the relation $\langle Au, v \rangle
= \langle u, Bv \rangle$, and we write $ B = A^*$. Here $(\cdot)^*$
denotes the Hermitian transpose, which in the case where $A$ is a real
matrix is just the transpose of $A$, $A^T$.  


Note that the gradient in
\eqref{eq:functional_gradient} can be rewritten (using the chain rule)
as
\begin{align}
  \nabla_{\mvec} \hat{I} =  \frac{\partial  I }{\partial \state} \nabla_{\mvec} \state
  + \frac{\partial  I }{\partial \mvec},
  \label{eq:functional_gradient_chain}
\end{align}
in which the $\dim(\mathbb{V}) \times \dim(\mathbb{Q})$ matrix
$\nabla_{\mvec} \state$ is difficult to compute.
Differentiating the force-balance equation \ref{} with respect to the
control parameters yields
\begin{align}
  & \nabla_{\mvec} \delta \Pi(\state, \mvec) = 0 \\
  \implies & \frac{\partial  \Pi }{\partial \state} \nabla_{\mvec} \state
  + \frac{\partial  \Pi }{\partial \mvec} = 0 \\
  \implies & \nabla_{\mvec} \state =
             - \left( \frac{\partial  \Pi }{\partial \state} \right)^{-1} \frac{\partial  \Pi }{\partial \mvec}.
\end{align}
Inserting this expression for $\nabla_{\mvec} \state$ into
\eqref{eq:functional_gradient_chain}, gives
\begin{align}
  \nabla_{\mvec} \hat{I} =  - \frac{\partial  I }{\partial \state}
  \left( \frac{\partial  \Pi }{\partial \state} \right)^{-1} \frac{\partial  \Pi }{\partial \mvec}
  + \frac{\partial  I }{\partial \mvec}.
\end{align}






%%% Local Variables:
%%% mode: latex
%%% TeX-master: "../../main"
%%% End:
