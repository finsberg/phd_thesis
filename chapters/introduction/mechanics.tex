\section{Mechanical modeling}

\subsection{Continuum description of the heart}

\begin{itemize}
  \item Reference to mehcanics book
  \item Introduce basic notation (stress, strain, deformation)
\end{itemize}

We represent the heart as a continuum body  embedded in
$\mathbb{R}^3$. Let $\Xvec$ and  $\xvec$ denote the corresponding
coordinate in the reference and current configuration respectively.
The motion of a point $\Xvec$ in the reference configuration to the
point $\xvec$ in the current configuration  may be described by a
deformation map  $\varphi :  \mathcal{B}_0  \rightarrow \mathcal{B}$.
The deformation gradient associated with the motion $\xvec =
\varphi(\Xvec)$ is a rank-2 tensor given by  
\begin{align}
\F(\Xvec) = \frac{\partial \varphi}{\partial \Xvec}, 
\end{align}
The deformed and reference configuration are related via the
displacement field $\uvec$ by: 
\begin{align}  
\uvec = \xvec-\Xvec = \chi( \Xvec, t) -\Xvec,
\end{align} 
Further, we let $J = \det \F$ be the determinant of the deformation gradient 
and $\C = \F^T\F$ the right Cauchy-Green deformation tensor.\cite{holzapfel2000nonlinear}




\subsection{Passive behavior of the myocardium}

\begin{itemize}
  \item Hyperelasticiy, fung type-refer to relevant papers
  \item Incompressibility
\end{itemize}



\subsection{Active behavior of the myocardium}


\begin{itemize}
  \item Active stress and strain (some mathematical aspects perhaps)
\end{itemize}



Saddle point problem

  

%%% Local Variables:
%%% mode: latex
%%% TeX-master: "../../main"
%%% End:
