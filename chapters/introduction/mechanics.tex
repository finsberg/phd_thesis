\section{Mechanical modeling}
\label{sec:intro_mechanical}

In this section we will ...
For a thorough review of continuum mehcanics we refer
to the book of Gerard Holzapel \cite{holzapfel2000nonlinear} from
which most of the theory in this section is taken. For the even more
mathematically oriented reader we refer to \cite{marsden1994mathematical}

\subsection{Introductory continuum mechanics}

\begin{itemize}
  \item Reference to mehcanics book
  \item Introduce basic notation (stress, strain, deformation)
  \end{itemize}


% The mechanical modeling of the heart is based on a continuum approach
% where the heart is represented as a continuum body  embedded in
% $\mathbb{R}^3$.

We represent the heart as a continuum body $\mathfrak{B}$ embedded in
$\mathbb{R}^3$. A configuration of $\mathfrak{B}$ is a mapping $\chi:
\mathfrak{B} \rightarrow \mathbb{R}^3$. 
We denote the \emph{reference configuration} of the heart by $\Omega_0
\equiv \chi_0(\mathfrak{B})$, and the \emph{current configuration} by $\Omega
\equiv \chi(\mathfrak{B})$. The mapping $\varphi :  \Omega_0
\rightarrow \Omega$ given by the composition $\varphi = \chi
\circ \chi_0^{-1}$ is a smooth, orientation preserving (posive
determinant) and invertible map. We denote the coordinates in the
referece configuration by $\Xvec \in \Omega_0$, and the coordinates in the current
configuration by $\xvec \in \Omega$. The coordinates $\Xvec$ and $\xvec$ are
commonly refered to as material and spatial points respectively, and
are related throgh the mapping $\varphi$, by $\xvec = \varphi(\Xvec)$.
For time-dependent problems it is common to make  the time-dependence
explicitely by writing $\xvec = \varphi(\Xvec, t)$. In the following
we will only focus on the mapping between two configurations and
therefore no time-dependence is needed. The mapping $\varphi$ will be
referred to as the \emph{motion}. The \emph{deformation gradient} is a
rank-2 tensor, defined as the partial derivative of the motion with
respect to the material coordinates:
\begin{align}
  \F(\Xvec) = \frac{\partial \varphi}{\partial \Xvec} = \Grad \xvec.
  \label{eq:deformation_gradient}
\end{align}
Here we also introduce the notation $\Grad$, which means derivative
with respect to reference coordinates.
The deformation gradient maps vectors in the referece configuration to
vectors in the current configuration, and belongs to the space of
linear transformations from $\mathbb{R}^3$ to $\mathbb{R}^3$ with
strictly posive determinant, which we denote by
$\mathrm{Lin}^+$. Another important quantity is the
\emph{displacement} field  
\begin{align}
  \uvec = \xvec-\Xvec, 
  \label{eq:displacement}
\end{align}
which relates postitions in the reference configuration to positions
in the current configuration. From \eqref{eq:deformation_gradient} we
see that
\begin{align}
  \F = \Grad \xvec = \Grad \uvec + \Grad \Xvec = \Grad \uvec + \I
\end{align}
Some other useful quantities are the \emph{right Cauchy-Green} deformation
tensor $\C = \F^T\F$, the \emph{left Cauchy-Green} deformation tensor
$\mathbf{B} = \F\F^T$, the \emph{Green-Lagrange} strain tensor
$\mathbf{E} = \frac{1}{2}(\C - \I)$, and the determinant of the
deformation gradient $J = \det \F$.

An important concept in mechanics is the concept of stress, which is
defined as force per area
$\left[\frac{\mathrm{N}}{\mathrm{m}^2}\right]$. When working with
different configurations one needs to be careful with which forces and
which areas we are talking about. Table \ref{tab:stress_tensor}
shows how forces and areas are related for the most important stress
tensors used in this thesis.  

\begin{table}[h]
  \centering
  \begin{tabular}{lll}
    \toprule
    Stress tensor & Forces & Area \\
    \midrule
    Second Piola-Kirchhoff ($\SPK$) & Referece configuration & Referece configuration \\
    First Piola-Kirchhoff ($\FPK$) & Current configuration  &  Referece configuration \\
    Cauchy ($\Cauchy$) &  Current configuration & Current configuration  \\
    \bottomrule
  \end{tabular}
  \caption{\label{tab:stress_tensor}Different stress tensor, and how
    they relate forces to areas though different configurations.}
\end{table}



\subsection{Hyperelasticity}

Even though experimental studies have idicated a viscoelastic behavior
of the myocardium \cite{dokos2002shear, gultekin2016orthotropic}, a
common assumption is to consider a quasistatic behaviour and therefore
it is also possible to model the myocardium as hyperelastic.
This means that we postulate the existence of
a strain-energy denisity function $\Psi:\mathrm{Lin}^+ \rightarrow
\mathbb{R}^+$ which is defined as energy per unit reference volume,
and has units $\frac{\text{Joule}}{m^3}$. The first Piola-Kirchhoff
stress for a hyperelastic material is given by th relation
\begin{align}
\FPK = \frac{\partial \Psi(\F)}{\partial \F}
\end{align}
The strain-energy density function provides us with the amount of
energy that is stored within the material in response to a given
strain. Hence, the stresses in a hyperelastic material with a given
strain-energy density function depends only on the strain, and not the
path for which the material deforms. On the contrary, if the model had
been viscoelastic we would expect to see hysterisis in the
stress/strain curve, but this is not possible for a hyperelastic
material. 

We say that the strain energy should be objective, meaning that the
stored energy in the material should be invariant with respect to
chage of observer. Formally we must have for any positive symmetric
rank-2 tensor $\C \in \mathrm{Sym}$ 
\begin{align}
  \Psi(\mathbf{C}) = \Psi(\mathbf{Q}\mathbf{C}\mathbf{Q}^T), \; \forall \mathbf{Q} \in \mathcal{G} \subseteq \mathrm{Orth},
\end{align}
where $\mathrm{Orth}$ is the group of all postive orthogonal matrices.
If $\mathcal{G} = \mathrm{Orth}$ we say that the matieral is
isotropic, and otherwise we say that the material is anisotropic.

\emph{In stead of introducing the invariants, look at Simones
  thesis. Introduce the invariants in Holzapel and Ogden model}


If the material is isotropic, the representation theorem for
invariants states that $\Psi$ can be expressed in terms of the principle
invariants of $\mathbf{C}$, theat is $\Psi = \Psi(I_1, I_2, I_3)$.
Here $I_i$ are the coefficients in the characteristic polynomial of
$\mathbf{C}$, and is given by 
\begin{align}
  I_1 = \tr \C,  && I_2 = \frac{1}{2}\left[ I_1^2 - \tr(\C^2)\right] && \text{and} && I_3 = \det \C.
\end{align}
In the case when the material constitutes a transversally isotropic
behavior, that is the material has a preferred direction $\mathbf{a}$
we have
\begin{align*}
  \mathcal{G} = \{ \mathbf{Q} \in \mathrm{Orth}: \mathbf{Q}(\mathbf{a}\otimes\mathbf{a})\mathbf{Q}^T\},
\end{align*}
with $\otimes$ being the outer product. In this case the strain-energy
denisity function can still be expressed through invariants. However,
we need to included the so called quasi-invariants, which are defined
as streches in the local microstructural coordinate-system. The
transversely isotropic invariants is given by
\begin{align}
  I_4 = \mathbf{a}_0 \cdot (\C \mathbf{a}_0) && \text{and} && I_5 = \mathbf{a}_0 \cdot (\C^2 \mathbf{a}_0).
\end{align}
For more detials about invariants see \cite{holzapfel2009constitutive,liu1982representations}.


Recent studies \cite{gultekin2016orthotropic} have raised doubts to
wether the myocardium can be considered purly hyperelastic, and
proposed a viscoelastic model for the myocardium. 

\subsection{Incompressibility}






\subsection{Balance laws and trasformations}

\subsection{Transformations between reference and current
  configuration}
By definition the reference configuration, $\Omega_0$ and current
configuration $\Omega$ are related via the motion $\varphi$ in the
sense that a point $p$ with reference coordinates $\Xvec$ and current
coordinates $\xvec$ satisfies $\xvec = \varphi(\Xvec)$. Likewise a
vector in the $\Omega_0$ is related to a vector in $\Omega$ via the
deformation gradient $\F$; if $\mathrm{d}\Xvec$ is a vector in the
reference configuration it will transform to the vector
$\mathrm{d}\xvec$ in the current cofiguration, and $\mathrm{d}\xvec =
\F \mathrm{d}\Xvec$. From this relation we also derrive that the
transformation of an infinitesimal volume element in the reference
cofiguration, $\mathrm{d}V$ is related to an infinitesimal volume
element in the current cofiguration, $\mathrm{d}v$  via the determinant of the
deformation gradient,
\begin{align}
  \mathrm{d}v =\det(\F) \mathrm{d}V.
  \label{eq:volume_element}
\end{align}
Another important transformation is the transformation of normal
vectors. By noting that we can write \eqref{eq:volume_element} using
surface elements
\begin{align*}
  \mathrm{d}s \mathbf{n} \mathrm{d}\xvec  &= \mathrm{d}v = \det(\F) \mathrm{d}V = \det(\F) \mathrm{d}S  \Nvec \mathrm{d}\Xvec\\
  &\implies \left( \mathrm{d}s \mathbf{n} \F  - \mathrm{d}S \det(\F) \Nvec \right) \mathrm{d}\Xvec = 0\\
  &\implies \left( \mathrm{d}s \F^T \mathbf{n}  - \mathrm{d}S \det(\F) \Nvec \right) \mathrm{d}\Xvec = 0,\\
\end{align*}
we get Nansens formula
\begin{align}
  \mathrm{d}s \mathbf{n}  =  \det(\F) \F^{-T} \mathrm{d}S \Nvec,
\end{align}
which related normal vector in the current configuration to normal
vector in the reference configuration.


% \subsubsection{Conservation of mass}
% Conservation of mass simply means that for a given motion $\varphi$
% from the reference configuration $\Omega_0$ to the current
% configuration $\Omega$ the mass is conserved. For a material with
% constant density the means that we have
% \begin{align}
%   \int_{\Omega_0} \mathrm{d} V = 
% \end{align}


\subsection{Conservation of linear momentum}


\subsection{Boudary Conditions}
\label{sec:mech_boudary}
Choosing the right boundary conditions for the model is essential,
and the choice should mimic what is observed in the real life. To
physiologically constrain the ventricle in a correct way is difficult,
and many different approaches has been proposed.
The boundary condition and the endocardium is typically modeled as a
Neumann boundary condition, representing the endocardial blood
pressure,
\begin{align}
  \mathbf{T} &= -p_{\mathrm{lv}}^{\mathrm{endo}} J\mathbf{F}^{-T} \cdot \mathbf{N}, \;  \mathbf{x} \in \lvendo
\end{align}
This condtion has a negative sign because the unit normal
$\mathbf{N}$ is pointing out of the domain, while the pressure is
acting into the domain. Likewise, it is common to enforce a
Neumann boundary condition on the epicardium,
\begin{align}
\mathbf{T} &= -p_{\mathrm{lv}}^{\mathrm{epi}} J\mathbf{F}^{-T} \cdot \mathbf{N}, \;  \mathbf{x} \in \epi.
\end{align}
However, the pressure $p_{\mathrm{lv}}^{\mathrm{epi}}$ is usually set
to zero. Therefore we drop the superscript and simply refere to the
left ventricular pressure and the left ventricular endocardial
pressure, i.e $p_{\mathrm{lv}} = p_{\mathrm{lv}}^{\mathrm{endo}}$.

The heart in enclosed in the fibroserous sac, called the pericardium
and slides within this sac during a heartbeat. A pericardial type
boundary constraint has been proposed in
\cite{nash2000computational}.

Another approach commonly used, is constrain th longitudinal motion of
base. To apply this type of condition, it is easiest if the base if
flat and located at a prescribed location, for example in the $x= 0$ plane. 



Using this
constrain alone is not enough since, the ventricle is free to move in
the basal plane.



Different
models of boundary conditons has been proposed in the litterature,
some begin more popular than others. In Figure we maker



One boundary conditions that is
not debatable is the boundary conditions on the endocardium. The 



in order to constrain you model appropratley at the boundary
and 

Write about different type of coundary conditions used, in partilar
fixing the base, completely, fixing the endoring, fixing the epiring



\subsection{Force-balance equation}
There are many ways to arrive at the force-balance eqations for the
cardiac mechanics problem. One way is consider conservation of
momentum, which arrives at Cauchys momentum equation. Since we
eventually want to solve this problem using the finite element method,
we only have to consider the equations in its weak form. An intuitive
way of doing this is by utilising a fundamental principle in physics
called the \emph{principle of stationary potential energy}, or
\emph{minimum total potential energy principle}, which states that a
physical system is at equillibrium when the total potential energy is
minmized, and any infinitesimal changes from this state should not add
any energy to the system.

We consider the incompressible case

In orther to make use of this principle we first need to add all the
potential energy in the system. Here we separate between internal and
external energy. Internal energy, is energy that is stored within the
material, for instance when you stretch a rubber band you increase its
internal energy. External energy is all

Denote the domain of interest by $\Omega \subset \mathbb{R}^3$, with
boundary $\partial \Omega$. For the case of a left ventrcular domain
we have $\partial \Omega = \lvendo \cup \lvepi \cup \lvbase$ and for a
bi-ventricular domain we include $\rvendo$ in the partition as well.


For a hyperelastic and incompressible materai, the total potential
energy in the system is give by


\begin{align}
  \Pi(\mathbf{u}, p) &= \Pi_{\mathrm{int}}(\mathbf{u},p) + \Pi_{\mathrm{ext}}(\mathbf{u}). \\
  \Pi_{\mathrm{int}}(u,p) &= \int_{\Omega} \left[ p(J(\mathbf{u}) - 1) +  \Psi(\mathbf{F}(\mathbf{u})) \right] \mathrm{d}V\\
  \Pi_{\mathrm{ext}}(\mathbf{u}) &= - \int_{\Omega} \mathbf{B} \cdot \mathbf{u} \mathrm{d} V - \int_{\partial \Omega_N} \mathbf{T} \cdot \mathbf{u} \mathrm{d}S
\end{align}

Here $\mathbf{B}$ represents body forces acting on a volume element in the reference domain,
e.g  gravity, and $\mathbf{T} = \mathbf{P} \mathbf{N}$ represents
first Piola-Kirchoff traction force acting on a surface element in the
referece domain, e.g pressure.
According to the \emph{Principle of stationary potential energy} we
have 
\begin{align}
  D_{\delta \mathbf{u}} \Pi(\mathbf{u}, p) = 0,  && \text{and} && D_{\delta p} \Pi(\mathbf{u}, p) = 0.
  \label{eq:minimum_potential_energy}
\end{align}
Here $\delta \mathbf{u}$ and $\delta p$ are virtual variations in the
space for the displacement and hydrostatic pressure respectively, and
\begin{align}
  D_{\mathbf{v}} \Phi(\mathbf{x}) = \frac{\mathrm{d}}{\mathrm{d}\varepsilon} \Phi(\mathbf{x} + \varepsilon \mathbf{v})\big|_{\varepsilon = 0}
\end{align}
is the directional derivative of $\Phi$ at $\mathbf{x}$ is the
direction $\mathbf{v}$. The operator is also known as the G\^ateaux
operator. The virtual variations $\delta \mathbf{u}$ and $\delta p$
represents an arbitrary direction with infinitesimal magnitude.


We have:
\begin{align}
  0 = D_{\delta p} \Pi(\mathbf{u}, p)
       % = \frac{\mathrm{d}}{\mathrm{d}\varepsilon} \int_{\Omega} \left(p + \varepsilon \delta p\right)(J(\mathbf{u}) - 1) \mathrm{d}V\\
  = \int_{\Omega}  \delta p(J(\mathbf{u}) - 1) \mathrm{d}V
\end{align}
and
\begin{align}
  \begin{split}
  0 = D_{\delta \mathbf{u}} \Pi(\mathbf{u}, p) %\\
  % =& \frac{\mathrm{d}}{\mathrm{d}\varepsilon} \int_{\Omega}  \left[ p(J(\mathbf{u} + \varepsilon \delta \mathbf{u}) - 1)
    % +  \Psi(\overline{\mathbf{F}}(\mathbf{u}+\varepsilon\delta \mathbf{u})) \right] \mathrm{d}V \\
     % &- \int_{\Omega} \mathbf{B} \cdot \delta \mathbf{u} \mathrm{d} V - \int_{\partial \Omega_N} \overline{\mathbf{T}} \cdot \delta \mathbf{u} \mathrm{d}S\\
  =&  \int_{\Omega}  \left[ pJ \mathbf{F}^{-T} + \mathbf{P} \right] : \Grad \delta \mathbf{u} \mathrm{d}V\\ 
  &- \int_{\Omega} \mathbf{B} \cdot \delta \mathbf{u} \mathrm{d} V - \int_{\partial \Omega_N} \overline{\mathbf{T}} \cdot \delta \mathbf{u} \mathrm{d}S
  \end{split}
\end{align}
We have the following boundary condition
\begin{align}
  u_1 &= 0 , \;  \mathbf{x} \in \lvbase \\
  k\mathbf{u} - \nabla \mathbf{u} \cdot  \mathbf{N}  &= \mathbf{0} , \;  \mathbf{x} \in \lvbase \\
  \mathbf{T} &= -p_{\mathrm{lv}} J\mathbf{F}^{-T} \cdot \mathbf{N}, \;  \mathbf{x} \in \lvendo.
\end{align}
This Dirichlet condition represent fixing the base in the longitudinal ($x$-direction)
direction, and is not present in the weak formuation.
The Robin conditon acts a spring term with sitffness $k \geq 0$ that limits the movement of the
base. The cornercase $k \rightarrow \infty$ represents fixing the base
in all directions while $k = 0$ means that the base are free to move
within the basal plane. The Neumann part of this conditon, $
\mathbf{T} = - \nabla \mathbf{u} \cdot  \mathbf{N}$ is a force. The
robin conditon  enforces that this force should increase linearly with
the displacement. Hence, a larger dispacement will impose a greater
force on the boundary. 
% The problem with this is that any 
The third conditon, i.e the Neumann condtion has a negative sign because the unit normal
$\mathbf{N}$ is pointing out of the domain, while the pressure is
acting into the domain. Plugging in the other two conditions we get the following
force balance equation:
\begin{align}
  \begin{split}
    0 
    % & \int_{\Omega}  \left[ pJ \mathbf{F}^{-T} + \mathbf{P} \right] : \nabla_{\mathbf{X}} \delta \mathbf{u} \mathrm{d}V \\
  % &+ \int_{\partial  \Omega_{i, \mathrm{LV endo}}} p_{\mathrm{lv}} J\mathbf{F}^{-T} \cdot \mathbf{N} \cdot \delta \mathbf{u} \mathrm{d}S\\
    % &- \int_{ \partial \Omega_{i, \mathrm{base}}} \nabla \mathbf{u} \cdot  \mathbf{N} \cdot \delta \mathbf{u} \mathrm{d}S \\
  =&  \int_{\Omega}  \left[ pJ \mathbf{F}^{-T} + \mathbf{P} \right] : \Grad \delta \mathbf{u} \mathrm{d}V \\
  &+ \int_{\partial  \Omega_{i, \mathrm{LV endo}}} p_{\mathrm{lv}} J\mathbf{F}^{-T} \cdot \mathbf{N} \cdot \delta \mathbf{u} \mathrm{d}S \\
  &- \int_{ \partial \Omega_{i, \mathrm{base}}} k \mathbf{u} \cdot \delta \mathbf{u} \mathrm{d}S
  \end{split}
\end{align}



\subsection{Discretization of the force balance equations}

The above eqations \eqref{} is only possible to solve analytically for
very simplified problems. Therefore we need to employ a numerial
approxation of the solution of the force balance equaiton. Using the
finite element method we often refer to such approximation as a
Garlerking approximation. This is based on approximating the solutions
by basis functions in a finite dimesional subspace of the true
solution. Let $(V, \langle \cdot \rangle_V)$
and $(Q, \langle \cdot \rangle_Q)$ be two suitable Hilbert
spaces for the displacement $\uvec$ and the hydrostatic pressure $p$
respectively. Now we select some finite dimesional subspaces $V_h
\subset V$ and $Q_h \subset Q$, which are spanned by a finite number
of basis functions.

For incompressible problems such as \eqref{}, we cannot choose the
approximation spaces $V_h, Q_h$ at random. A known numerical issue
that arises is locking, which can be seen if the material do not
deform even if forces are applied. To illustrate the problem of
locking , imagine an incomressible sphere exsposed to higher and higer
hydrostatic pressure. In other words, the hydrostatic pressure is not
unique for a given diplacement. The problem is solved by requirig the
finte element approximation to satisfy the discrete inf-sup condition
\cite{arnold1984stable}. There exist several mixed elements that
satistifies the inf-sup condition. The one used in this thesis is the
Taylor-Hood elements \cite{taylor1973numerical}, with


\subsection{Constitutve relations}



\subsubsection{Modeling of the passive myocardium}
\begin{itemize}
  \item Hyperelasticiy, fung type-refer to relevant papers
  \item Incompressibility - Saddle point problem
  \item Finding the unloaded configuration
\end{itemize}


Talk about the Holzapel Odgen Law, the invariants, the othotropic one,
transvarsally isotropic one, the meaning of each term, analytic
version of the stress

\subsubsection{Modeling of the active contraction}


\begin{itemize}
  \item Active stress and strain (some mathematical aspects perhaps)
\end{itemize}

  

One property that separate the myocardium from other materials is its
ability to generate force without external stimulation. As discussed
in Sction \ref{some part from electrophysiology}, then myocardium
contracts in a cyclic manner blabal.

There are basically two main approaches to model the active response
of the myocardium; the \emph{active stress} and the \emph{active strain}
formulation.

In the active stress formulation \cite{hunter1998modelling} one assumes
that the total Cauchy stress $\Cauchy$ can be written as an additive sum of one
passive contribution $\Cauchy_p$ and one active contribution $\Cauchy_a$:
\begin{align}
  \Cauchy = \Cauchy_p + \Cauchy_a && \Cauchy_p = J^{-1} \frac{\partial \Psi(\F)}{\partial \F} \F^{T}
\end{align}


Historically the active stress formulation has been the


\subsubsection{Choosing the reference geometry}
A general problem in the biomechanics is that geometries extrated from
imaging data if not the stress-free geometry, meaning the the geometry
that we observe is subjected to a physical load, e.g blood pressure.
This means that 


\subsection{Implementation details}
A word abour fenics and how we can implement this
Newtons method, good initial guess, increament pressure.
Relatively small system, dense matrix, plot stiffness matrix, Direct
solver.
Fenics compilation stored in cache.
Fibers at quadrature points. Symbolic language.

\subsubsection{FEniCS}



\subsubsection{Numerical considerations}
The solution of non-linear problems such as the one described here is
typically solved using methods like Newton's method. The convergece of
such methods depends on the initial guess, and if the
initial guess is too far from the solution, the solver might diverge.
Moreover, if the initial guess is close to the solution the
convergence rate is in general quadratic.

Let us consider a
typical numerical problem of inflating the ventricular geomtry from a
stress-free configuration to end-diastole. This involves increasing
the pressure, or the boundary tranction on the endocardium, from zero
to the end-diasolic pressure. A strategy know as the
\emph{incremental load} technique is usually a good approach. In this
strategy you select some incremental step-size (for instance $0.4$
kPa), and increase the pressure linearly until the target pressure is
reached. If the solver diverges you decrease the step-size (for
instance by a factor of 0.5) until convergence is reached, and
continue to step up the pressure with the new step-size. This is very
robust, but definetly a slow approach. Since many of the 
constitutive models for myocardium consist of an exponential
realtionship between the stress and strain (so call Fung-type
relation), the amount of stress needed to displace a material will be
higher if the matieral is a state with high strain compared to a state
of low strain. Therefore, the Newtons solver might perform less
iterations to reach convergence when when the load is increased. As a
result, one could improve the incremental load technique by adapting
the step sice if the number of newton iterations are below a certain
threshold (for instance $8$ iterations).

An even more clever strategy, uses a technique from bifurcation
chaos theory and is known as numerical continuation
\cite{allgower2003introduction}.  Suppose we want to
solve the non-linear problem $F(\uvec, \lambda)=0$ with state variable
$\uvec$ and parameter $\lambda$. For instance $\uvec$ could be the
displacement and $\lambda$ could be the endocardial pressure.
The idea be numerical continuations is that given a solution pair
$(\uvec_0, \lambda_0)$ there exist (under conditons stated by the
implicit function theorem) a solution curve $\uvec(\lambda)$ such that
$F(\uvec(\lambda), \lambda)=0$ and $\uvec(\lambda_0) = \uvec_0$.
To explicitly find such a curve is not allways easy but a simple
approximation can be found by extrapolation: Given two pairs
$(\uvec_0, \lambda_0)$ and $(\uvec_1, \lambda_1)$, and a new target
parameter $\lambda_2$, a possible solution is 
\begin{align}
  \uvec_2 =  (1-\delta)\uvec_0 + \delta \uvec_1 && \delta = \frac{\lambda_2 - \lambda_0}{\lambda_1 - \lambda_0}.
\end{align}
% Let $w_0$ denote the initial
% state variable associated with endocardial pressure $p_0$. Increase
% the pressure $p_1 = p_0 + \Delta p_0$, and solve to obtain $w_1$. Next
% we would like to solve for $p_2 = p_1 + \Delta p_1$ where $\Delta
% p_1$ might be the adapted step size. In stead of using $\tilde{w_2}=w_1$ as
% initial guess for the newton solver, as we typically would do in the incremental load
% technique, we observe that if $\delta = \frac{p_2 - p_0}{p_1 - p_0}$,
% and hence $p_2 = (1-\delta)p_0 + \delta p_1$, then a better choice
% of intial guess would be $\tilde{w_2} = (1-\delta)w_0 + \delta w_1$.
Choosing $\uvec_2$ as initial guess for the non-linear solver has been
successfully performed by others in non-linear caridac mechanics
problems \cite{pezzuto2013mechanics}. 


\subsection{Parall Performance}
Write a section here where we test a sample code on multiple cores to
test scalability




%%% Local Variables:
%%% mode: latex
%%% TeX-master: "../../main"
%%% End: