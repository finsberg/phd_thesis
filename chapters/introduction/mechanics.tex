\section{Mechanical modeling}
\label{sec:intro_mechanical}

In this section we will cover the necessary theory of continuum
mechanics in order to model the mechanics of the heart. The theory of
continuum mechanics is extensive, and we will not be able to cover
everything. For a complete review of continuum mechanics the reader is
therefore referred the textbook of Gerard Holzapfel
\cite{holzapfel2000nonlinear} from which most of the theory in this
section is taken. For the even more mathematically oriented reader we
refer to \cite{marsden1994mathematical}.

We represent the heart as a continuum body $\mathfrak{B}$ embedded in
$\mathbb{R}^3$. A configuration of $\mathfrak{B}$ is a mapping $\chi:
\mathfrak{B} \rightarrow \mathbb{R}^3$. 
We denote the \emph{reference configuration} of the heart by $\Omega
\equiv \chi_0(\mathfrak{B})$, and the \emph{current configuration} by $\omega
\equiv \chi(\mathfrak{B})$. The mapping $\varphi :  \Omega
\rightarrow \omega$ given by the composition $\varphi = \chi
\circ \chi_0^{-1}$ is a smooth, orientation preserving (positive
determinant) and invertible map. We denote the coordinates in the
reference configuration by $\Xvec \in \Omega$, and the coordinates in the current
configuration by $\xvec \in \omega$. The coordinates $\Xvec$ and $\xvec$ are
commonly referred to as material and spatial points respectively, and
are related through the mapping $\varphi$, by $\xvec = \varphi(\Xvec)$.
For time-dependent problems it is common to make  the time-dependence
explicitly by writing $\xvec = \varphi(\Xvec, t)$. In the following
we will only focus on the mapping between two configurations and
therefore no time-dependence is needed. The mapping $\varphi$ will be
referred to as the \emph{motion}. The \emph{deformation gradient} is a
rank-2 tensor, defined as the partial derivative of the motion with
respect to the material coordinates:
\begin{align}
  \F(\Xvec) = \frac{\partial \varphi}{\partial \Xvec} = \Grad \xvec.
  \label{eq:deformation_gradient}
\end{align}
Here we also introduce the notation $\Grad$, which means derivative
with respect to reference coordinates.
The deformation gradient maps vectors in the reference configuration to
vectors in the current configuration, and belongs to the space of
linear transformations from $\mathbb{R}^3$ to $\mathbb{R}^3$ with
strictly positive determinant, which we denote by
$\mathrm{Lin}^+$. Another important quantity is the
\emph{displacement} field  
\begin{align}
  \uvec = \xvec-\Xvec, 
  \label{eq:displacement}
\end{align}
which relates positions in the reference configuration to positions
in the current configuration. From \eqref{eq:deformation_gradient} we
see that
\begin{align}
  \F = \Grad \xvec = \Grad \uvec + \Grad \Xvec = \Grad \uvec + \I
\end{align}
Some other useful quantities are the \emph{right Cauchy-Green} deformation
tensor $\C = \F^T\F$, the \emph{left Cauchy-Green} deformation tensor
$\mathbf{B} = \F\F^T$, the \emph{Green-Lagrange} strain tensor
$\mathbf{E} = \frac{1}{2}(\C - \I)$, and the determinant of the
deformation gradient $J = \det \F$.

An important concept in mechanics is the concept of stress, which is
defined as force per area
$\left[\frac{\mathrm{N}}{\mathrm{m}^2}\right]$. When working with
different configurations one needs to be careful with which forces and
which areas we are talking about. Table \ref{tab:stress_tensor}
shows how forces and areas are related for the most important stress
tensors used in this thesis. Note that the explicit form of the stress
tensor requires a consitutive law for the material at hand. This will
be discussed in more detail in Section \ref{sec:constitutive_relations}.

\begin{table}[h]
  \centering
  \begin{tabular}{lll}
    \toprule
    Stress tensor & Forces & Area \\
    \midrule
    Second Piola-Kirchhoff ($\SPK$) & Reference configuration & Reference configuration \\
    First Piola-Kirchhoff ($\FPK$) & Current configuration  &  Reference configuration \\
    Cauchy ($\Cauchy$) &  Current configuration & Current configuration  \\
    \bottomrule
  \end{tabular}
  \caption{\label{tab:stress_tensor}Showing different stress tensors
    used in this thesis, and how
    they relate forces to areas trough different configurations.}
\end{table}


\subsection{Balance laws and transformations}
In this section we will cover some basic transformation used to
derrive the fore-balance equations for the mechanics of the heart. 

\subsubsection{Transformations between reference and current
  configuration}
By definition the reference configuration, $\Omega$ and current
configuration $\omega$ are related via the motion $\varphi$ in the
sense that a point $\mathfrak{p}$ with reference coordinates $\Xvec$ and current
coordinates $\xvec$ satisfies $\xvec = \varphi(\Xvec)$. Likewise a
vector in the $\Omega$ is related to a vector in $\omega$ via the
deformation gradient $\F$; if $\mathrm{d}\Xvec$ is a vector in the
reference configuration it will transform to the vector
$\mathrm{d}\xvec$ in the current configuration, and $\mathrm{d}\xvec =
\F \mathrm{d}\Xvec$. From this relation we also derive that the
transformation of an infinitesimal volume element in the reference
configuration, $\mathrm{d}V$ is related to an infinitesimal volume
element in the current configuration, $\mathrm{d}v$  via the determinant of the
deformation gradient,
\begin{align}
  \mathrm{d}v =\det(\F) \mathrm{d}V.
  \label{eq:volume_element}
\end{align}
Another important transformation is the transformation of normal
vectors. By noting that we can write \eqref{eq:volume_element} using
surface elements
\begin{align*}
  \mathrm{d}s \mathbf{n} \mathrm{d}\xvec  &= \mathrm{d}v = \det(\F) \mathrm{d}V = \det(\F) \mathrm{d}S  \Nvec \mathrm{d}\Xvec\\
  &\implies \left( \mathrm{d}s \mathbf{n} \F  - \mathrm{d}S \det(\F) \Nvec \right) \mathrm{d}\Xvec = 0\\
  &\implies \left( \mathrm{d}s \F^T \mathbf{n}  - \mathrm{d}S \det(\F) \Nvec \right) \mathrm{d}\Xvec = 0,\\
\end{align*}
we get \emph{Nanson's formula}
\begin{align}
  \mathrm{d}s \mathbf{n}  =  \det(\F) \F^{-T} \mathrm{d}S \Nvec,
\end{align}
which relates the normal vector in the current configuration to the
normal vector in the reference configuration.


\subsubsection{Conservation of linear momentum}
Newton's seconds law states that the change in linear momentum equals
the net impulse acting on it. For a continuum material with constant
mass density $\rho$ this implies that
\begin{align}
  \int_{\omega} \rho \dot{\mathbf{v}} \mathrm{d}v = \mathbf{f},
  && \mathbf{f} = \int_{\partial \omega} \mathbf{t} \mathrm{d}s
     + \int_{\omega} \mathbf{b} \mathrm{d}v,
     \label{eq:cons_lin_mom}
\end{align}
where $\mathbf{v}$ is the spatially velocity field, $\mathbf{t}$ is
the traction acting on the boundary, and $\mathbf{b}$ is the body
force. From \emph{Cauchy's stress theorem} we know that there exist a
second order tensor $\sigma$, known as as Cauchy stress tensor that is
related to the traction vector by $\mathbf{t} = \sigma \mathbf{n}$,
where $\mathbf{n}$ is the unit normal in the current configuration.
Using the divergence theorem we get.

\begin{align*}
  \int_{\partial \Omega} \mathbf{t} \mathrm{d}s
  = \int_{\partial \Omega} \sigma \mathbf{n} \mathrm{d}s
  = \int_{\Omega} \nabla \cdot \sigma \mathrm{d}v,
\end{align*}
and by collecting the terms from \eqref{eq:cons_lin_mom}. We arrive at
Cauchys momentum equation
\begin{align}
  \nabla \cdot \sigma + \mathbf{b} =  \rho \dot{\mathbf{v}}.
  \label{eq:chauch_momentum_eq}
\end{align}
The contribution for the body force ($\mathbf{b}$)  and inertial term
($\rho \dot{\mathbf{v}}$) are negligible compared to the stresses
\cite{hunter1996kd,tallarida1970left, moskowitz1981effects}, which is
why the force balance equations is typically only stated as
\begin{align}
  \nabla \cdot \sigma = \mathbf{0}.
  \label{eq:momentum_simple_current}
\end{align}
Note that we have formulated the balance law in the current
configuration. An equivalent statement can be formulated in terms of
the reference configuration
\begin{align}
  \nabla \cdot \FPK = \mathbf{0}, 
  \label{eq:momentum_simple_reference}
\end{align}
where $\FPK$ is the first Piola-Kirchhoff stress tensor.

\subsubsection{Conservation of angular momentum}
Just like linear momentum, the angular momentum is also a conserved
quantity. We will not go through the derivation of resulting
equations, but state that as a consequence, the Cauchy stress tensor is
symmetric
\begin{align}
  \sigma = \sigma^T.
\end{align}



\subsection{Hyperelasticity}
\label{sec:hyperelasticity}

Even though experimental studies have indicated a visco-elastic behavior
of the myocardium \cite{dokos2002shear, gultekin2016orthotropic}, a
common assumption is to consider a quasi-static behavior and therefore
it is also possible to model the myocardium as hyperelastic.
This means that we postulate the existence of
a strain-energy density function $\Psi:\mathrm{Lin}^+ \rightarrow
\mathbb{R}^+$, and that stress is given by the relation
\begin{align}
\FPK = \frac{\partial \Psi(\F)}{\partial \F}.
\end{align}
Since stress has unit Pa, we see that the strain-energy density
function is defined as energy per unit reference volume, and has units
$\frac{\text{Joule}}{m^3}$. 
The strain-energy density function relates  the amount of
energy that is stored within the material in response to a given
strain. Hence, the stresses in a hyperelastic material with a given
strain-energy density function, depends only on the strain, and not the
path for which the material deforms. On the contrary, if the model had
been visco-elastic we would expect to see hysteresis in the
stress/strain curve, but this is not possible for a hyperelastic
material. 

\begin{remark}
  The second law of thermodynamics states that the total entropy
  production in a thermodynamic process can never be nagative. Elastic
  materials defines a special class of materials in which the entropy
  production is zero. Within this thermodynamic framework the
  strain-energy density function coincides (up to a constant) with the
  Helmholtz free energy density.
\end{remark}


\subsubsection{General requirements for the strain-energy density
  function}
\label{sec:strain_energy_req}
Some general requirements must hold for the strain-energy function.
First of all, we require that the reference state is stress free and
that the stored energy increases monotonically with the deformation. 
Formally this can be stated simply as 
\begin{align*}
  \Psi(\I) = 0 && \Psi(\F) \geq 0.
\end{align*}
Moreover, if expanding or compressing a body to zero volume would
require an infinite amount of energy:
\begin{align*}
  \Psi(\F) \rightarrow \infty \; && \text{as} &&\; \det \F \rightarrow& 0 \\
  \Psi(\F) \rightarrow \infty \; && \text{as} &&\; \det \F \rightarrow& \infty
\end{align*}
We say that the strain energy should be objective, meaning that the
stored energy in the material should be invariant with respect to
change of observer. Formally we must have: \emph{given any positive symmetric
rank-2 tensor $\C \in \mathrm{Sym}$:}
\begin{align}
  \Psi(\mathbf{C}) = \Psi(\mathbf{Q}\mathbf{C}\mathbf{Q}^T), \; \forall \mathbf{Q} \in \mathcal{G} \subseteq \mathrm{Orth}.
\end{align}
Here $\mathrm{Orth}$ is the group of all positive orthogonal matrices.
If $\mathcal{G} = \mathrm{Orth}$ we say that the material is
isotropic, and otherwise we say that the material is anisotropic.
This brings us to another important issue, which is related to the
choice of coordinate-system. Having to deal with different
coordinate-systems, and mapping quantities from one coordinate-system
to another can results in complicated computations. Therefore it would be beneficial if we
could work with quantities which do not depend on the choice of
coordinate-system. Such quantities are called invariants. 
If the material is isotropic, the representation theorem for
invariants states that $\Psi$ can be expressed in terms of the
principle invariants of $\mathbf{C}$, that is $\Psi = \Psi(I_1, I_2,
I_3)$. The principle invariants $I_i, i=1,2,3$ are the coefficients in
the characteristic polynomial of $\mathbf{C}$, and is given by 
\begin{align}
  I_1 = \tr \C,  && I_2 = \frac{1}{2}\left[ I_1^2 - \tr(\C^2)\right] && \text{and} && I_3 = \det \C.
\end{align}
In the case when the material constitutes a transversely isotropic
behavior, that is the material has a preferred direction $\mathbf{a}_0$,
which in the case of the myocardium could be the direction of fiber
muscle fibers, we have
\begin{align*}
  \mathcal{G} = \{ \mathbf{Q} \in \mathrm{Orth}: \mathbf{Q}(\mathbf{a}_0\otimes\mathbf{a}_0)\mathbf{Q}^T\},
\end{align*}
with $\otimes$ being the outer product. In this case the strain-energy
density function can still be expressed through invariants. However,
we need to included the so called quasi-invariants, which are defined
as stretches in the local microstructural coordinate-system. The
transversely isotropic invariants is given by
\begin{align*}
  I_{4\mathbf{a}_0 } = \mathbf{a}_0 \cdot (\C \mathbf{a}_0) && \text{and} && I_5 = \mathbf{a}_0 \cdot (\C^2 \mathbf{a}_0).
\end{align*}
The invariants do have a physical interpretation. For instance, $I_3$
is related to the volume ratio of material during deformation, while
$I_{4\mathbf{a}_0 } $ is related to the stretch along the direction
$\mathbf{a}_0 $. Indeed the \emph{stretch} ratio in the direction
$\mathbf{a}_0$ is given by $\lambda_{\mathbf{a}_0} = | \F \mathbf{a}_0
|$ and we see that $I_{4\mathbf{a}_0 }  =  \mathbf{a}_0 \cdot (\C
\mathbf{a}_0) = \F \mathbf{a}_0 \cdot (\F \mathbf{a}_0) =
\lambda_{\mathbf{a}_0}^2$. For more details about invariants see
\cite{holzapfel2009constitutive,liu1982representations}.


The theory of global existence of unique solutions for elastic problem
was originally based convexity of the free energy function.
A function $\phi: \Omega \rightarrow \mathbb{R}$ is \emph{convex} if for any
$\Xvec_1, \Xvec_2 \in \Omega$ we have
\begin{align}
  \phi(\lambda \Xvec_1+ (1-\lambda) \Xvec_2)
  \leq \lambda \phi(\Xvec_1)
  + (1-\lambda) \phi(\Xvec_2).
\end{align}
However, from a physical point of view this requirement is to strict
\cite{ball1976convexity}. A slightly weaker requirement is that the
strain-energy function $\Psi$ is \emph{polyconvex}, meaning that there exist
a convex function $\phi$ such that
\begin{align*}
  \Psi(\F) = \phi(\F, \cof \F, \det \F).
\end{align*}
Note however, that if $\Psi$ is convex then it is also polyconvex.


\subsection{Incompressibility}
\label{sec:incompressibility}
An incompressible material is a material in which only isochoric
motions are possible. This means that volume of the material do not
change during any deformation and hence we have the constraint
\begin{align}
  J = \det(\F) = 1.
  \label{eq:incompressible_cons}
\end{align}
The constraint \eqref{eq:incompressible_cons} can be imposed by considering the strain energy
function
\begin{align}
  \Psi = \Psi(\F) + p(J-1),
  \label{eq:incomp_strain_energy}
\end{align}
where $p$ is a scalar which serves as a Lagrange multiplier, but which
can be identified as the hydrostatic pressure. Like the displacement,
the hydrostatic pressure is unknown and has to be determined from the equilibrium
equations \eqref{eq:force_balance_strong_ref}. If we differentiate
\eqref{eq:incomp_strain_energy} with respect to $\F$ we get the First
Piola-Kirchhoff stress tensor for an incompressible material
\begin{align}
  \FPK = \frac{\partial \Psi(\F)}{\partial \F} + J p \F^{-T}.
\end{align}
Likewise the Cauchy stress tensor is given by
\begin{align}
  \Cauchy = J^{-1} \frac{\partial \Psi(\F)}{\partial \F}\F^{T} + p \I.
  \label{eq:cauchy_incomp}
\end{align}


\begin{remark}
  The sign of $p$ is determined by whether you add or subtract the term
  $ p(J-1)$ to the total strain energy in
  \eqref{eq:incomp_strain_energy}. For all practical purposes, it
  does not matter if you add or subtract the term as long as you are
  consistent. 
\end{remark}


\subsubsection{Uncoupling of volumetric and isochoric response}
The total strain energy function in \eqref{eq:incomp_strain_energy}
is written as a sum of isochoric and volumetric components. Let
\begin{align}
  \F =  \F_{\mathrm{vol}} \F_{\mathrm{iso}},
\end{align}
then $ \F_{\mathrm{vol}} =
J^{1/3}\I$ and $\F_{\mathrm{iso}} = J^{-1/3}\F$. For
compressible material (i.e with $J \neq 1$) it is important to consider
only deviatoric strains in the strain-energy density function, so that
$\Psi = \Psi_{\mathrm{iso}}(\F_{\mathrm{iso}}) +
\Psi_{\mathrm{vol}}(J)$. For incompressible material ($J = 1$), we
have $\F_{\mathrm{vol}} = \I$ so that such a decomposition seems
uneccesary. However, a similar decomposition has shown to be
numerically beneficial \cite{weiss1996finite}. Note that, in this case, a similar
decoupling of the stress tensors has to be done.

\subsection{Boundary Conditions}
\label{sec:mech_boudary}


Choosing the right boundary conditions for the model is essential,
and the choice should mimic what is observed in the reality. To
physiologically constrain the ventricle in a correct way is difficult,
and many different approaches has been proposed.
The boundary condition and the endocardium is typically modeled as a
Neumann boundary condition, representing the endocardial blood
pressure
\begin{align}
  \sigma \mathbf{n} = -p_{\mathrm{lv}} \mathbf{n}, \;  \xvec \in  \lvendo(t).
\end{align}
This condition has a negative sign because the unit normal
$\mathbf{N}$ is pointing out of the domain, while the pressure is
acting into the domain. 
Note that this condition is imposed on the current configuration, and
to utilize the Lagrangian formulation we can pull back this condition
to the reference configuration to obtain
\begin{align}
  \FPK\mathbf{N} &= -p_{\mathrm{lv}}^{\mathrm{endo}} J\mathbf{F}^{-T} \cdot \mathbf{N}, \;  \Xvec \in \lvendo
\end{align}
Likewise, it is common to enforce a
Neumann boundary condition on the epicardium,
\begin{align}
\FPK\mathbf{N}  &= -p_{\mathrm{lv}}^{\mathrm{epi}} J\mathbf{F}^{-T} \cdot \mathbf{N}, \;  \Xvec \in \epi.
\end{align}
However, the pressure $p_{\mathrm{lv}}^{\mathrm{epi}}$ is usually set
to zero. Therefore we drop the superscript and simply referred to the
left ventricular pressure as the left ventricular endocardial
pressure, i.e $p_{\mathrm{lv}} = p_{\mathrm{lv}}^{\mathrm{endo}}$.

There exist a variety of boundary conditions at the base.
It is common to constrain the longitudinal motion of
base, even though it is observed in cardiac images that the apex tend
to be more fixed than the base. A recent study shows that taking into
account the base movement is important to capture the correct
geometrical shape \cite{palit2016passive}. However, this has not been
done in the studies in this thesis.
Fixing the longitudinal motion at the base is enforced through a
Dirichlet boundary condition
\begin{align}
  u_1 = 0,  \;  \Xvec \in \lvbase
\end{align}
To apply this type of condition, it is easiest if the base is
flat and located at a prescribed location, for example in the $x= 0$
plane.
Constraining the longitudinal motion of the base alone is not enough,
since the ventricle is free to move in the basal plane. In order to
anchor the geometry it is possible to fix the movement of the base in
all directions
\begin{align}
  \uvec = \mathbf{0},  \;  \Xvec \in \lvbase,
\end{align}
or fixing the endocardial or epicardial ring
\begin{align}
  \uvec &= \mathbf{0},  \;  \Xvec \in \Gamma_{\mathrm{endo}} \\
  \uvec &= \mathbf{0},  \;  \Xvec \in \Gamma_{\mathrm{epi}}.
\end{align}
Another approach which is used in this thesis is to impose a Robin
type boundary condition at the base
\begin{align}
  \FPK \Nvec + k \uvec = \mathbf{0},  \;  \Xvec \in \lvbase, 
\end{align}
or at the epicardium to mimic the pericardium
\begin{align}
  \FPK \Nvec + k \uvec = \mathbf{0},  \;  \Xvec \in \lvepi, 
\end{align}
Here $k$ can be seen as the stiffness of a spring that limit the
movement. The limiting cases, $k = 0$ and $k \rightarrow
\infty$ represent free and fixed boundary respectively.
More complex boundary conditions to mimic the pericardium is also
possible \cite{fritz2014simulation}, but not considered in this thesis.



\subsection{Force-balance equation}
We will now collect all the terms that makes up to force balance for
the cardiac mechanics problem. Considering the myocardium as a incompressible,
hyperelastic material with a Robin type boundary condition at the base
we obtain the following set of equations in the Lagrangian formulation
\begin{align}
  \begin{split}
  \nabla \cdot \FPK &= 0 \\
  J - 1 &= 0,
  \end{split}
 \label{eq:force_balance_strong}
\end{align}
with $J = \det(\F)$, and appropriate boundary conditions.
% \begin{align}
%   \FPK \Nvec + k \uvec &= \mathbf{0},  \;  \Xvec \in \lvbase\\
%   u_1 &= 0,  \;  \Xvec \in \lvbase \\
%   \FPK\mathbf{N}  &= \mathbf{0}, \;  \Xvec \in \epi \\
%   \FPK\mathbf{N} &= -p_{\mathrm{lv}} J\mathbf{F}^{-T} \cdot \mathbf{N}, \;  \Xvec \in \lvendo
%   \label{eq:bndry_conditions_strong}
% \end{align}
 


\subsubsection{Variational formulation}
\label{sec:variational_formulation}
There are many ways to arrive at the variational formulation of the
force-balance equations for the cardiac mechanics problem.  One way is
to consider the strong form in \eqref{eq:force_balance_strong} and use
the standard approach in the finite element method to multiply by
test function in a suitable space, and perform integration by
parts. Within the fields of continuum mechanics it is common to
refer to this approach as the \emph{principle of virtual work}, with states
that the virtual work of all forces applied to a mechanical system
vanishes in equilibrium. Within this framework, test functions are
referred to as virtual variations.
Another approach, which we will use here, derives the variational form
by utilizing a fundamental principle in physics
called the \emph{principle of stationary potential energy}, or
\emph{minimum total potential energy principle}. This principle states that a
physical system is at equilibrium when the total potential energy is
minimized, and any infinitesimal changes from this state should not add
any energy to the system.
In order to make use of this principle we first need to sum up all the
potential energy in the system. Here we separate between internal and
external energy. Internal energy, is energy that is stored within the
material, for instance when you stretch a rubber band you increase its
internal energy. External energy represent the contribution from all
external forces such as gravity and traction forces.
% We consider the equilibrium in the referece domain, and denote the
% domain of interest by $\Omega \subset \mathbb{R}^3$, with boundary
% $\partial \Omega$. For the case of a left ventrcular domain we have
% $\partial \Omega = \lvendo \cup \lvepi \cup \lvbase$ and for a
% bi-ventricular domain we include $\rvendo$ in the partition as well. 
For a hyperelastic and incompressible material, the total potential
energy in the system is given by
\begin{align}
  \Pi(\mathbf{u}, p) &= \Pi_{\mathrm{int}}(\mathbf{u},p) + \Pi_{\mathrm{ext}}(\mathbf{u}). \\
  \Pi_{\mathrm{int}}(u,p) &= \int_{\Omega} \left[ p(J(\mathbf{u}) - 1) +  \Psi(\mathbf{F}(\mathbf{u})) \right] \mathrm{d}V\\
  \Pi_{\mathrm{ext}}(\mathbf{u}) &= - \int_{\Omega} \mathbf{B} \cdot \mathbf{u} \mathrm{d} V - \int_{\partial \Omega_N} \mathbf{T} \cdot \mathbf{u} \mathrm{d}S
\end{align}
Here $\mathbf{B}$ represents body forces acting on a volume element in
the reference domain, e.g  gravity, and $\mathbf{T} = \mathbf{P}
\mathbf{N}$ represents first Piola-Kirchhoff traction force acting on
the Neumann boundary $\partial \Omega_N$. According to the
\emph{Principle of stationary potential energy} we have  
\begin{align}
  D_{\delta \mathbf{u}} \Pi(\mathbf{u}, p) = 0,  && \text{and} && D_{\delta p} \Pi(\mathbf{u}, p) = 0.
  \label{eq:minimum_potential_energy}
\end{align}
Here $\delta \mathbf{u}$ and $\delta p$ are virtual variations in the
space for the displacement and hydrostatic pressure respectively, and
\begin{align}
  D_{\mathbf{v}} \Phi(\mathbf{x}) = \frac{\mathrm{d}}{\mathrm{d}\varepsilon} \Phi(\mathbf{x} + \varepsilon \mathbf{v})\big|_{\varepsilon = 0}
\end{align}
is the directional derivative of $\Phi$ at $\mathbf{x}$ is the
direction $\mathbf{v}$. The operator is also known as the G\^ateaux
operator. The virtual variations $\delta \mathbf{u}$ and $\delta p$
represents an arbitrary direction with infinitesimal magnitude. We have
\begin{align}
  0 = D_{\delta p} \Pi(\mathbf{u}, p)
  = \int_{\Omega}  \delta p(J(\mathbf{u}) - 1) \mathrm{d}V,
\end{align}
and
\begin{align*}
  \begin{split}
  0 = D_{\delta \mathbf{u}} \Pi(\mathbf{u}, p) 
  =&  \int_{\Omega}  \left[ pJ \mathbf{F}^{-T} + \mathbf{P} \right] : \Grad \delta \mathbf{u} \mathrm{d}V - \int_{\Omega} \mathbf{B} \cdot \delta \mathbf{u} \mathrm{d} V
  \end{split}
\end{align*}
Note that the traction forces are now incorporated into the stress
tesors after application of the divergence theorem. These equations
are called the Euler-Lagrange equations. Here $\uvec  \in V = 
H_D^1(\Omega) = \{ \mathbf{v}: \int_{\Omega} \left[ |\mathbf{v}|^2 +  |\Grad
\mathbf{v}|^2 \right]\mathrm{dV} < \infty \wedge \mathbf{v}\big|_{\partial
  \Omega_D} = 0\}$ and $p \in Q = L^2(\Omega)$, with $\partial
\Omega_D$ representing the Dirichlet boundary.
In sumarry, the Euler-Lagrange equations written in a mixed form reads
: \emph{Find $(\uvec, p)\in V \times Q$ such that}
\begin{align}
  \begin{pmatrix}
    D_{\delta p} \Pi(\mathbf{u}, p)\\
    D_{\delta \mathbf{u}} \Pi(\mathbf{u}, p) 
  \end{pmatrix}
  = \mathbf{0}.  && \forall \; (\delta \uvec, p) \in V \times Q.
\end{align}


\subsubsection{Discretization of the force balance equations}

Equation \eqref{eq:force_balance_strong} is only possible to solve
analytically for very simplified problems. Therefore we need to employ
a numerical approximation method to solve the problem. One such method
is the finite element method (FEM). When using the finite element method we often refer to such
approximation as a Galerkin approximation. This is based on
approximating the solution by linear combinations of  basis functions in a finite dimensional
subspace of the true solution. Let $(V, \langle \cdot \rangle_V)$
and $(Q, \langle \cdot \rangle_Q)$ be two suitable Hilbert
spaces for the displacement $\uvec$ and the hydrostatic pressure $p$
respectively (see Section \ref{sec:variational_formulation}). Now we
select some finite dimensional subspaces $V_h \subset V$ and $Q_h
\subset Q$, which are spanned by a finite number of basis functions.


For incompressible problems such as \eqref{eq:force_balance_strong}, we cannot choose the
approximation spaces $V_h, Q_h$ at random. A known numerical issue
that arises for such saddle-point problems is \emph{locking}, which can be
seen if the material do not deform even if forces are applied. The
problem is solved by requiring the finite element approximation to
satisfy the discrete inf-sup condition \cite{le1982existence}. There
exist several mixed elements that satisfies this condition
\cite{chapelle1993inf}. The elements used in this thesis are the
Taylor-Hood finite elements \cite{taylor1973numerical}. Let the domain of
interest be denoted by $\Omega$, and let $\mathcal{T}_h$ be a
triangulation of that domain in the sense that $\bigcup_{T \in
  \mathcal{T}_h} T = \overline{\Omega}$. Denote by $\mathcal{P}_k
(T)$, the linear space of all polynomials of degree $\leq k$ defined
on $T$. Then for $k \geq 2$, the Taylor-Hood finite elements are the spaces
\begin{align}
  V_h = \{  \phi \in C(\Omega) \;  | \; \phi \big| T  \in \mathcal{P}_k (T) , T \in \mathcal{T}_h \},\\
  Q_h = \{  \phi \in C(\Omega) \;  | \; \phi \big| T  \in \mathcal{P}_{k-1} (T) , T \in \mathcal{T}_h \},
\end{align}
where $C(\Omega)$ denotes the space of continuous function on
$\Omega$. In this thesis we have exclusively used these elments with
$k = 2$.



\begin{remark}
  The basis functions that span the Taylor-Hood
  finite element spaces are also known as the
  Lagrangian basis functions. These basis functions, of
  degree $n$, are a polynomials of degree $n$ on each element, but only
  continuous at the nodes (i.e not continuously
  differentiable). Consequently, differentiating a function that is
  expressed as a linear combination of the Lagrangian basis functions,
  will not be continuous at the nodes, and therefore caution has to
  be made when evaluating features that depends on the derivative of
  such functions. Examples of such features are stress and
  strain with depends on the deformation gradient which again depends
  on the derivative of the displacement. One way to deal with this
  issue is to 1) use other types of elements that are continusly differenetible
  everywhere,  such a the cubic Hermite elements or 2) evaluate the features at the
  Gaussian quadrature points where there is no problem with continuity.
\end{remark}


\subsection{Constitutive relations}
\label{sec:constitutive_relations}
We have now covered a general mechanical framework which holds for
material science in general. What differentiate the mechanics of soft
living tissue, like the myocardium, from other materials is the
constitutive relations which describes the response of a material to
applied load. Such constitutive relations often comes from
experimental observations, both on observation of anatomical structure
but also from experiments done of tissue slabs.

We have already covered the theory of hyperelasticity and incompressibility in Section
\ref{sec:hyperelasticity} and \ref{sec:incompressibility} respectively
which are types of constitutive relations. In this section we will
cover consitutive relations which only apply to soft living tissue
such as the myocardium. 

A complete constitutive model of the mechanical behavior of the
myocardium must account for both the passive and the active response
of the myocardium.


\subsubsection{Modeling of the passive myocardium}

The passive response of the myocardium has been investigated through
uni-axial, bi-axial and shear deformation experiments \cite{dokos2002shear}.
% Look at Humprey Continuum biomechanics of
% soft biological tissues
In 2009 Holzapfel and Ogden proposed an orthotropic constitutive model
of the passive myocardium \cite{holzapfel2009constitutive} which is
based on the experiments done in \cite{dokos2002shear}. The model
assumes a local orthonormal coordinate system with the fiber axis
$\ef$, sheet axis $\eS$ and sheet-normal axis $\en$. %The subscript
% refers that the vectors $(\ef, \eS, \en)$ are vectors in the reference
% configuration.
From this coordinate system we define the invariants
\begin{align}
  \begin{split}
    I_1 &= \tr(\C) ,\\
    I_{4\ef} &= \ef \cdot (\C \ef),\\
    I_{4\eS} &= \eS \cdot (\C \eS),\\
    I_{8\ef\eS} &=  \eS \cdot (\C \ef), 
  \end{split}
\end{align}
Here $I_{4\ef} $ and $I_{4\eS}$ are the stretches along the
fiber, sheet axis respectively and $I_{8\ef\eS}$ is
related to the angle between the fiber and sheets in the current
configuration given that they are orthogonal in the reference
configuration. Note that since $(\ef, \eS, \en)$ is an orthonormal
system, we have the relation $I_1 = I_{4\ef} + I_{4\eS} +I_{4\en}$,
and so $I_{4\en}$ is redundant. The orthotropic Holzapfel and Ogden
model reads
\begin{align}
  \label{eq:holzapel_full}
  \begin{split}
  \Psi(I_1, I_{4\ef},  I_{4\eS},  I_{8\ef\eS}) =& \frac{a}{2 b} \left( e^{ b (I_1 - 3)}  -1 \right)\\
  +& \frac{a_f}{2 b_f} \left( e^{ b_f (I_{4\ef} - 1)_+^2} -1 \right) \\
  +& \frac{a_s}{2 b_s} \left( e^{ b_s (I_{4\eS} - 1)_+^2} -1 \right)\\
  +& \frac{a_{fs}}{2 b_{fs}} \left( e^{ b_{fs} I_{8\ef\eS}^2} -1 \right).
\end{split}
\end{align}
Here $( x )_+ = \frac{1}{2} \left( x + |x| \right)$, so that the
the terms involving $I_{4\ef}$ and $I_{4\eS}$ only contribute to the
stored energy during elongation. From \eqref{eq:holzapel_full} we see
that it is easy to identify the physical meaning of each term. For
example the first term represents the isotropic contribution which is
the overall stiffness in the extracellular matrix while the second
term accounts for the extra stiffness along the fibers when they are
elongated. It is also straight forward to prove that the strain-energy
function is convex, and that the requirements for existence and
uniqueness discussed in Section \ref{sec:strain_energy_req} is
fulfilled.

In this thesis we have in several occasions used a
transversely isotropic version of\eqref{eq:holzapel_full} which is
obtained by setting $a_{fs} = b_{fs}= a_s = b_s = 0$, i.e
\begin{align}
  \label{eq:holzapel_trans}
  \begin{split}
  \Psi(I_1, I_{4\ef}) = \frac{a}{2 b} \left( e^{ b (I_1 - 3)}  -1 \right)
  + \frac{a_f}{2 b_f} \left( e^{ b_f (I_{4\ef} - 1)_+^2} -1 \right).
  \end{split}
\end{align}
If we further set $a_f = b_f = b = 0$ so that in $a$ is the
only nonzero parameter, then the Holzapfel-Ogden model reduces to (after a
series expansion of the exponential and a limiting argument)
\begin{align}
  \Psi(I_1)  = \frac{a}{2} \left( I_1 - 3 \right), 
\end{align}
which is the model of a Neo Hookean material. The Cauchy stress can be derived
analytically from \eqref{eq:holzapel_full}, by using the chain rule and
\eqref{eq:cauchy_incomp}, 
\begin{align}
  \begin{split}
    \Cauchy
    =& \frac{\partial \Psi(\F)}{\partial \F}\F^{T} + p \I
    = \sum_{i \in \left\{ 1, 4\ef,  4\eS,  8\ef\eS \right\} }
    \psi_i \frac{\partial I_i}{\partial \F}\F^{T} + p \I \\
    =& p \I + a \left( e^{ b (I_1 - 3)}  -1 \right) \mathbf{B} 
    + 2 a_f (I_{4\ef} - 1)_+  e^{ b_f (I_{4\ef} - 1)^2} \mathbf{f} \otimes \mathbf{f} \\
    &+ 2 a_f (I_{4\eS} - 1)_+  e^{ b_f (I_{4\eS} - 1)^2} \mathbf{s} \otimes \mathbf{s} 
    + a_{fs} I_{8\ef\eS}  e^{ b_{fs} I_{8\ef\eS}^2} \left( \mathbf{f} \otimes \mathbf{s}  +  \mathbf{s} \otimes \mathbf{f} \right)
  \end{split}
\end{align}


\subsubsection{Modeling of the active contraction}
  

One feature that separate the myocardium from other hyperelastic
materials such as rubber, is its ability to actively generate force
without external loads. This active 

As discussed 
in Section \ref{sec:ventricular_pumping_function}, the myocardium
contracts in a cyclic manner. 

There are basically two main approaches to model the active response
of the myocardium; the \emph{active stress} and the \emph{active strain}
formulation.


Based one the classical three element Hill model illustrated in Figure
\ref{fig:hill_muscle_model}, the active contribution naturally
decomposes the total stress into a sum of passive and active stresses
\cite{nash2004electromechanical}. Hence, in the active stress
formulation \cite{hunter1998modelling} one assumes that the total
Cauchy stress $\Cauchy$ can be written as an additive sum of one
passive contribution $\Cauchy_p$ and one active contribution 
$\Cauchy_a$;
\begin{align}
  \Cauchy = \Cauchy_p + \Cauchy_a && J \Cauchy_p =  \frac{\partial \Psi(\F)}{\partial \F} \F^{T}
\end{align}
The passive contribution is determined by the material model used,
while the active contribution is given by 
\begin{align}
  \Cauchy_a = \Cauchy_{ff} \Fef \otimes \Fef +
  \Cauchy_{ss} \mathbf{s} \otimes \mathbf{s} +
  \Cauchy_{nn} \mathbf{n} \otimes \mathbf{n},
\end{align}
and the different constants are typically coupled to the
electrophysiology and calcium dynamics.
There are experimental evidence that the transverse active stresses
are non-negligible \cite{lin1998multiaxial}. One approach is to assume
a uniform transverse avtivation in which the total active tension
can be written as 
\begin{align}
  \Cauchy_a = T_a \left[\Fef \otimes \Fef +
   \eta\left( \mathbf{s} \otimes \mathbf{s} +
  \ \mathbf{n} \otimes \mathbf{n} \right)\right],
\end{align}
where $\eta$ represent the amount of transverse activation and $T_a
\in \mathbb{R}$ is the magnitude of the active tension.
In the limiting case ($\eta = 0.0$), the active tension acts purely
along the fibers and is reduced to 
\begin{align}
  \Cauchy_a = T_a \Fef \otimes \Fef.
\end{align}
Note that, by observing  that
\begin{align*}
  \frac{\partial I_{4\mathbf{a}_0}}{\partial \F} = \frac{\partial (\mathbf{a}_0  \cdot \C \mathbf{a}_0 )}{\partial \F}
  = 2 \mathbf{a} \otimes \mathbf{a}_0 \implies  \mathbf{a} \otimes  \mathbf{a}= \frac{1}{2} \frac{\partial I_{4\mathbf{a}_0}}{\partial \F} \F^{T}
\end{align*}
and that $I_1 =  I_{4\ef} +  I_{4\eS} +  I_{4\en}$, 
we can instead decompose the strain-energy into a passive and active
part \cite{pathmanathan2010cardiac}, $\Psi= \Psi_p + \Psi_a$, with
\begin{align}
\Psi_a = \frac{T_a}{2J} \left(( I_{4\ef} - 1)  + \eta \left[ (I_1 - 3) -
    (I_{4\ef} - 1)\right] \right), 
\end{align}
so that $J \Cauchy_a  = \frac{\partial \Psi_a}{\partial \F} \F^{T}$.
The active strain formulation is a relatively new way of modeling the
active contraction in the heart and was first introduced in
\cite{taber2000modeling}. This formulation is based on a
multiplicative decomposition of the deformation gradient.  
\begin{equation}
 \F = \F_e \F_a.
\label{eq:active_strain}
\end{equation}
The 

The general form of the active deformation gradient for an orthotropic
material with active response is given by
\begin{equation}
  \F_a =  \I
  - \gamma_f \ef \otimes \ef
  - \gamma_s \eS \otimes \eS
  - \gamma_n \en\otimes \en
 \label{eq:active_strain_Fa_general}
\end{equation}

We add the constraint $\det(\F_a) = 1$, meaning that the active
deformation is volume preserving. Further we assume that the activation is
transversely isotropic, so that the sheet and sheet-normal axis is
treated in the same way. Then  we see that $\gamma_n = \gamma_s =1-
(1-\gamma_f)^{-1/2}$, and we have
\begin{equation}
  \F_a = (1 - \gamma) \ef \otimes \ef  + \frac{1}{\sqrt{1 - \gamma}} (\I - \ef \otimes \ef), 
 \label{eq:active_strain_Fa_gjerald}
\end{equation}
where we set $\gamma = \gamma_f$ for convenience. 


While the motivation behind the active stress formulation is purely
physiological and based on the Hill model, the motivation behind the
active strain formulation is more driven by ensuring mathematical
robustness. In particular it has been shown \cite{ambrosi2012active}
that with the active strain approach properties such as frame
invariance and rank-one ellipticity is inherited from the strain
energy of the material. In contrast, rank-one elipticity is not
guaranteed for the active stress formulation.

The active stress approach is by far the most used model of active
contraction and is based on the classical Hill model, consisting of a
three element model with one contractile element 

Hybrid version \cite{goktepe2014generalized}




\subsection{Implementation details}
A word about FEniCS and how we can implement this
Newtons method, good initial guess, increment pressure.
Relatively small system, dense matrix, plot stiffness matrix, Direct
solver.
FEniCS compilation stored in cache.
Fibers at quadrature points. Symbolic language.

\subsubsection{FEniCS}
The FEniCS project started out in 
FEniCS is a framework for solving partial differential
equations using the finite element method. FEniCS has a python
interface where the user can write code which is close to the
mathematical notions used in the theory.

code written in
FEniCS is close to the mathematical notation, and the handling of
assembly code happens ``under the hood'', which makes it easy to use
and intuitive for scientists 

\subsubsection{Numerical considerations}
The solution of non-linear problems such as the one described here is
typically solved using methods like Newton's method. The convergence of
such methods depends on the initial guess, and if the
initial guess is too far from the solution, the solver might diverge.
Moreover, if the initial guess is close to the solution the
convergence rate is in general quadratic.

Let us consider a
typical numerical problem of inflating the ventricular geometry from a
stress-free configuration to end-diastole. This involves increasing
the pressure, or the boundary traction on the endocardium, from zero
to the end-diastolic pressure. A strategy know as the
\emph{incremental load} technique is usually a good approach. In this
strategy you select some incremental step-size (for instance $0.4$
kPa), and increase the pressure linearly until the target pressure is
reached. If the solver diverges you decrease the step-size (for
instance by a factor of 0.5) until convergence is reached, and
continue to step up the pressure with the new step-size. This is very
robust, but definitely a slow approach. Since many of the 
constitutive models for myocardium consist of an exponential
relationship between the stress and strain (so call Fung-type
relation), the amount of stress needed to displace a material will be
higher if the material is a state with high strain compared to a state
of low strain. Therefore, the Newtons solver might perform less
iterations to reach convergence when when the load is increased. As a
result, one could improve the incremental load technique by adapting
the step since if the number of newton iterations are below a certain
threshold (for instance $8$ iterations).

An even more clever strategy, uses a technique from bifurcation
chaos theory and is known as numerical continuation
\cite{allgower2003introduction}.  Suppose we want to
solve the non-linear problem $F(\uvec, \lambda)=0$ with state variable
$\uvec$ and parameter $\lambda$. For instance $\uvec$ could be the
displacement and $\lambda$ could be the endocardial pressure.
The idea be numerical continuations is that given a solution pair
$(\uvec_0, \lambda_0)$ there exist (under conditions stated by the
implicit function theorem) a solution curve $\uvec(\lambda)$ such that
$F(\uvec(\lambda), \lambda)=0$ and $\uvec(\lambda_0) = \uvec_0$.
To explicitly find such a curve is not always easy but a simple
approximation can be found by extrapolation: Given two pairs
$(\uvec_0, \lambda_0)$ and $(\uvec_1, \lambda_1)$, and a new target
parameter $\lambda_2$, a possible solution is 
\begin{align}
  \uvec_2 =  (1-\delta)\uvec_0 + \delta \uvec_1 && \delta = \frac{\lambda_2 - \lambda_0}{\lambda_1 - \lambda_0}.
\end{align}
% Let $w_0$ denote the initial
% state variable associated with endocardial pressure $p_0$. Increase
% the pressure $p_1 = p_0 + \Delta p_0$, and solve to obtain $w_1$. Next
% we would like to solve for $p_2 = p_1 + \Delta p_1$ where $\Delta
% p_1$ might be the adapted step size. In stead of using $\tilde{w_2}=w_1$ as
% initial guess for the newton solver, as we typically would do in the incremental load
% technique, we observe that if $\delta = \frac{p_2 - p_0}{p_1 - p_0}$,
% and hence $p_2 = (1-\delta)p_0 + \delta p_1$, then a better choice
% of intial guess would be $\tilde{w_2} = (1-\delta)w_0 + \delta w_1$.
Choosing $\uvec_2$ as initial guess for the non-linear solver has been
successfully performed by others in non-linear cardiac mechanics
problems \cite{pezzuto2013mechanics}. 





%%% Local Variables:
%%% mode: latex
%%% TeX-master: "../../main"
%%% End: