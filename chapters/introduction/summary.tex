
\section{Summary of papers}


\subsection{Paper 1}

\subsection{Paper 2}

\subsection{Paper 3}

\subsection{Paper 4}


\section{Other contributions}
Along with the research articles presented in this thesis, other
types of contributions in terms of talks, posters and software has
been made during the writing of this thesis. These contributions are
listed below.

\subsection{Talks}
\begin{itemize}
  \item Henrik Finsberg, Gabriel Balaban, Joakim Sundnes, Hans
    Henrik Odland, Marie Rognes, and Samuel T. Wall. ``Patient
    Constrained Ventricular Stress Mapping'',
    Conference Presentation at MALT 2015,  Lugano, Switzerland (2015).
  \item Henrik Finsberg, Gabriel Balaban, Joakim Sundnes, Marie
    Rognes, and Samuel T. Wall. ``Personalization of a Cardiac
    Compuational Model using Clinical Measurements'', Conference
    Presentation at 28th Nordic Seminar on Computational
    Mechanics. Vol. 28. Tallin, Estonia, (2015).
  \item Henrik Finsberg, Gabriel Balaban, Joakim Sundnes, Marie
    Rognes, and Samuel T. Wall. ``Optimization of a Spatially Varying
    Cardiac Contraction parameter using the Adjoint Method'',
    Conference Presentation at FEniCS 16, Oslo, Norway,(2016).
  \item Finsberg, Henrik N., Gabriel Balaban, Joakim Sundnes, Hans
    Henrik Odland, Marie Rognes, and Samuel T. Wall. ``Personalized
    Cardiac Mechanical Model using a High Resolution Contraction Field
    '',  Conference Presentation at VPH16 Translating VPH to the
    Clinic,  Amsterdam, Netherlands (2016).
\end{itemize}


\subsection{Posters}
\begin{itemize}
  \item Henrik Finsberg, Gabriel Balaban, Joakim Sundnes, Marie
    Rognes, and Samuel T. Wall. ``Patient Specific Modeling of Cardiac
    Mechanics using the Active Strain Formulation '',
    Geilo Winter School, Geilo, Norway, (2016).
  \item Henrik Finsberg, Ce Xi, J. Tan, L. Zhong, LC Lee, Joakim
    Sundnes, and Samuel T. Wall. ``Mechanical Analysis of Pulmonary
    Hypertension via Adjoint based Data Assimilation of a Finite
    Element Model '', Summer Biomechanics, Bioengineering, and
    Biotransport Conference, Tucson, AZ, (2017). 
  \end{itemize}


\subsection{Software}
  \urlstyle{rm}
\begin{itemize}
  \item Pulse-Adjoint, FEniCS-based cardiac mechanics solver and data
    assimilator, source: \url{https://bitbucket.org/finsberg/pulse_adjoint}
  \item Mesh-Toolbox, Toolbox for generating FEniCS meshes from 4D
    Echo,  source: \url{https://bitbucket.org/finsberg/mesh_generation}
\end{itemize}



\section{Closing remarks and future directions}

A couple remarks regarding the limitations of the methods used in this
thesis shoud be made. Here we start by listing a couple of statements
which should be taken into considerations. 


\begin{itemize}
\item \emph{The quality of features/biomarkers you can extract from a data-driven
model cannot be any better than the data used as input to the
model}. Most of the results presented in this thesis are based on data
obtained from clinical measurements of real patients.
\item \emph{The spatial resolution of the parameters should be
    reflected in the spatial resolution of the observations.} When
  trying to fit data that are spatially resolved at some level,
  choosing parameters that are resolved at a finer level should be
  done with caution. The continuity in the underlying physics as well
  as regularization techniques could be used to ...

\end{itemize}

During the work of this thesis, several questions still remains open
and would require

\begin{itemize}
\item The active model for the myocardium is has .. Degree of
  tranverse activation. Matching of strain data..
\item Identifiability of parameters... Uniqueness of
  solutions.. Amount of regularization.. Convexity of the mismatch functional
\item Appropriate boundary conditions.. In paper three we saw big
  differences in the choice of boundary conditions. Especially, the
  magnitude of the stress seems to 
\item Coupling of electrophysiologigy and mechanics in an
  ajoint-based. data assimilation framework. 
\end{itemize}

%%% Local Variables:
%%% mode: latex
%%% TeX-master: "../../main"
%%% End:
