
\section{Summary of papers}
In this final introductory section we summarize the papers that make
up this thesis, and we also give some final concluding remarks
and future directions. 

\subsection{Paper 1: High-resolution data assimilation of cardiac mechanics
  applied to a dyssynchronous ventricle}
In this paper we develop and test the pipeline for constructing a
patient specific mechanical simulation of a patient's heart, based on
clinical measurments and adjoint-based data assimilation
techniques.

As a model case we consider one patient, diagnosed with
left bundle branch block and selected for cardiac resynchronization
therapy (CRT). Prior to the CRT implantation the patient had 4D
echocardiography taken for which the LV geometry, LV volumes and LV
regional strains throughout the cardiac cycle are measured. During
implantation of the CRT device the LV pressure where also measured
invasively. The LV pressure measurements are used as boundary condition at the
endocardium, while the LV volume and LV regional strain is incorporated
into a cost functional that we want to minimize.

The pipeline is divided into two phases, a passive phase where we
estimate the linear isotropic parameters $a$ in
\eqref{eq:holzapel_trans} as a global material parameter using the
measurement points belonging atrial systole, and a active phase where we estimate the
a spatially varying contraction parameter ($\gamma$ in
\eqref{eq:intro_active_strain_Fa_gjerald}) at each measurement point
with active contraction. During the passive phase we only fit the
volumes, while during the active phase 51 strain measurements in the radial,
longitudinal and circumferential direction in each AHA segment (Figure
\ref{fig:echopac_output}) are used in the optimization. 

The results show an excellent fit with measured strain an volume, with
a average relative error in the volume and strain of less than 0.4 \% and 3 \%
respectively using a contraction parameter with one degree of freedom
for each vertex in the geometry (2661 parameters in total). Parameters
at lower spatial resolution are also tested to show the necessity of
high spatial resolution to fit the measured strain data.
A synthetic test is also performed in order to show that that method
is able to recapitulate the generated data, also with noise added to
the data. Moreover, a sensitivity analysis to different parameters are
presented in the appendix.


\subsection{Paper 2: Estimating cardiac contraction through high resolution
  data assimilation of a personalized mechanical model}
In this paper we apply the method developed in the previous paper to a
cohort of patients and estimate indices of cardiac contractility. More
specifically, a group of seven patients diagnosed with left bundle branch block and
selected for cardiac resynchronization therapy (CRT) and a group of
seven healthy control subjects where included in the study.

The pipeline is similar to the one outlined in paper 1, with the
exception that we also estimate the unloaded, stress-free
configuration using the method outlined in Section
\ref{sec:intro_coupled_material}. 

The optimized active strain parameter in
\eqref{eq:intro_active_strain_Fa_gjerald} as well as the optimized
active stress parameter in \eqref{eq:intro_active_stress} are averaged
over the ventricle and compared between the two groups. The healthy
group showed a significant increase in both of these
parameters. Furthermore, estimation of end-systolic elastance by
perturbation of the pressure at the end-systolic state while fixing
the remaining quantities were made. This estimate of end-systolic
elastance where also significantly higher in the healthy control
group. 


\subsection{Paper 3: Mechanical Analysis of pulmonary hypertension via
  adjoint based data assimilation of a finite element model.}

\subsection{Paper 4: Assesment region regional myocardial work using
  adjoint-based data assimilation}

\newpage
\section{Other contributions}
Along with the research articles presented in this thesis, other
types of contributions in terms of talks, posters and software has
been made during the writing of this thesis. These contributions are
listed below.

\subsection{Talks}
\begin{itemize}
  \item Henrik Finsberg, Gabriel Balaban, Joakim Sundnes, Hans
    Henrik Odland, Marie Rognes, and Samuel T. Wall. ``Patient
    Constrained Ventricular Stress Mapping'',
    Conference Presentation at MALT 2015,  Lugano, Switzerland (2015).
  \item Henrik Finsberg, Gabriel Balaban, Joakim Sundnes, Marie
    Rognes, and Samuel T. Wall. ``Personalization of a Cardiac
    Compuational Model using Clinical Measurements'', Conference
    Presentation at 28th Nordic Seminar on Computational
    Mechanics. Vol. 28. Tallin, Estonia, (2015).
  \item Henrik Finsberg, Gabriel Balaban, Joakim Sundnes, Marie
    Rognes, and Samuel T. Wall. ``Optimization of a Spatially Varying
    Cardiac Contraction parameter using the Adjoint Method'',
    Conference Presentation at FEniCS 16, Oslo, Norway,(2016).
  \item Finsberg, Henrik N., Gabriel Balaban, Joakim Sundnes, Hans
    Henrik Odland, Marie Rognes, and Samuel T. Wall. ``Personalized
    Cardiac Mechanical Model using a High Resolution Contraction Field
    '',  Conference Presentation at VPH16 Translating VPH to the
    Clinic,  Amsterdam, Netherlands (2016).
\end{itemize}


\subsection{Posters}
\begin{itemize}
  \item Henrik Finsberg, Gabriel Balaban, Joakim Sundnes, Marie
    Rognes, and Samuel T. Wall. ``Patient Specific Modeling of Cardiac
    Mechanics using the Active Strain Formulation '',
    Geilo Winter School, Geilo, Norway, (2016).
  \item Henrik Finsberg, Ce Xi, J. Tan, L. Zhong, LC Lee, Joakim
    Sundnes, and Samuel T. Wall. ``Mechanical Analysis of Pulmonary
    Hypertension via Adjoint based Data Assimilation of a Finite
    Element Model '', Summer Biomechanics, Bioengineering, and
    Biotransport Conference, Tucson, AZ, (2017). 
  \end{itemize}


\subsection{Software}
\begin{itemize}
  \item Pulse-Adjoint, FEniCS-based cardiac mechanics solver and data
    assimilator, source: \url{https://bitbucket.org/finsberg/pulse_adjoint}
  \item Mesh-Toolbox, Toolbox for generating FEniCS meshes from 4D
    Echo,  source: \url{https://bitbucket.org/finsberg/mesh_generation}
\end{itemize}


\newpage
\section{Closing remarks and future directions}

Although we have shown in this thesis that
adjoint-based data assimilation is a powerful technique that opens new
possibilities in terms of patient specific mechanical simulations, many questions
still has to be answered before we can fully embrace the output of
such simulations.

First of all, is should be clear that \emph{the
  quality of biomarkers you can extract from a data-driven
  model cannot be any better than the data used as input to the
  model}. A typical saying is that garbage in $=$ garbage out,
meaning that if the data you use to constrain the model is noisy, then
you will also fit this noise if you allow for enough degree of
freedom. Regularization techniques (Section
\ref{sec:intro_regularization}) provides a way to attack this problem,
but it is not clear what is the best approach. If the noise in the
data is normally distributed with zero expected value, then adding
more data to the cost functional will also have a regularizing effect.
Hence, as the data assimilation techniques developed in this thesis
can take into account large amount of data, as much data as possible
should be provided to the assimilator.

A rule of thump is that \emph{the spatial resolution of the parameters should be
  reflected in the spatial resolution of the observations}. This means
that if one is trying to fit data that are spatially resolved at some level,
then choosing parameters that are resolved at a finer level should be
done with caution. Regarding both paper 1 and 2 we see that the
spatial resolution chosen was at a much finer level than the input
data. In this case, regularization techniques was used to restrict the
parameter space. 

When estimating high dimensional parameters, a natural question
concerning uniqueness of these estimates arises. It is obvious that if one allows for
enough degree of freedom in the parameter space, then it is possible
to fit almost any type of data. Therefore \emph{more work on ensuring
identifiability} of these estimates (Section
\ref{sec:intro_identifiability}) should be done. If the output of such
model should have any clinical utility, then uniqueness of the
estimated parameters absolutely pivotal. Moreover, \emph{validation} of
these model is what matter most in terms of translating such computational
models into the clinic.

Regarding the mechanical modeling of the heart, 
\emph{choosing appropriate boundary conditions} that reflect the reality has
been a problem during the work in this thesis, and several different
choices has been made. Moreover, \emph{accounting for the orthotropic as
well as the visco-elastic behavior of the myocardium} is something that should
be investigated in future studies. Accounting for an orthotropic
behavior would acquire more parameters to be estimated, which would
thus require more input data.

In this thesis we have also not fully
explored the \emph{spatial resolution of the material parameters}, and it
should be investigated whether it is possible to relate locally estimated
tissue stiffness to e.g myocardial infarction.

The choice of active model for the myocardium is another topic that
should be addressed in more detail in future studies. In the author's
opinion, the output of the two fundamentally different approaches may
vary a lot. In particular, it is evident that the active strain
formulation do a better job in fitting strain data. One
hypothesis is that the amount of transverse active stresses should be
adjusted to each individual. These transverse active stresses are
naturally embedded in the active strain approach via a volume
preserving active deformation. 

Finally, coupling the electrophysiology to the mechanical model within
this adjoint based data assimilation framework is something that
should also be investigated in the future. 

% Here we start by listing a couple of statements
% which should be taken into considerations. 


% \begin{itemize}
% \item \emph{The quality of features/biomarkers you can extract from a data-driven
% model cannot be any better than the data used as input to the
% model}. Most of the results presented in this thesis are based on data
% obtained from clinical measurements of real patients.
% \item \emph{The spatial resolution of the parameters should be
%     reflected in the spatial resolution of the observations.} When
%   trying to fit data that are spatially resolved at some level,
%   choosing parameters that are resolved at a finer level should be
%   done with caution. The continuity in the underlying physics as well
%   as regularization techniques could be used to ...

% \end{itemize}

% During the work of this thesis, several questions still remains open
% and would require

% \begin{itemize}
% \item The active model for the myocardium is has .. Degree of
%   tranverse activation. Matching of strain data..
% \item Identifiability of parameters... Uniqueness of
%   solutions.. Amount of regularization.. Convexity of the mismatch functional
% \item Appropriate boundary conditions.. In paper three we saw big
%   differences in the choice of boundary conditions. Especially, the
%   magnitude of the stress seems to 
% \item Coupling of electrophysiologigy and mechanics in an
%   ajoint-based. data assimilation framework. 
% \end{itemize}

%%% Local Variables:
%%% mode: latex
%%% TeX-master: "../../main"
%%% End:
