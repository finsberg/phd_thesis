
\section{Summary of papers}


\subsection{Paper 1}

\subsection{Paper 2}

\subsection{Paper 3}

\subsection{Paper 4}

\section{Closing remarks and future directions}

A couple remarks regarding the limitations of the methods used in this
thesis shoud be made. Here we start by listing a couple of statements
which should be taken into considerations. 


\begin{itemize}
\item \emph{The quality of features/biomarkers you can extract from a data-driven
model cannot be any better than the data used as input to the
model}. Most of the results presented in this thesis are based on data
obtained from clinical measurements of real patients.
\item \emph{The spatial resolution of the parameters should be
    reflected in the spatial resolution of the observations.} When
  trying to fit data that are spatially resolved at some level,
  choosing parameters that are resolved at a finer level should be
  done with caution. The continuity in the underlying physics as well
  as regularization techniques could be used to ...

\end{itemize}

During the work of this thesis, several questions still remains open
and would require

\begin{itemize}
\item The active model for the myocardium is has .. Degree of
  tranverse activation. Matching of strain data..
\item Identifiability of parameters... Uniqueness of
  solutions.. Amount of regularization..
\item Appropriate boundary conditions.. In paper three we saw big
  differences in the choice of boundary conditions. Especially, the
  magnitude of the stress seems to 
\item Coupling of electrophysiologigy and mechanics in an
  ajoint-based. data assimilation framework. 
\end{itemize}
 


%%% Local Variables:
%%% mode: latex
%%% TeX-master: "../../main"
%%% End:
